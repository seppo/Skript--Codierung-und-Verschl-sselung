\chapter{Symmetrische Blockchiffre}
\section{Blockchiffre}
Zerlege Klartext in Bl�cke (Strings) der L�nge $n$.  Jeder Block wird einzeln verschl�sselt (in der Regel wieder in einem Block der L�nge $n$). Gleiche Bl�cke werden gleich verschl�sselt.\\
Wieviele Blockchiffren der L�nge $n$ gibt es?\\
Alphabet $\Z_2=\{0,1\}$\\
$|\{\underbrace{(0,\ldots,0)}_{Block},(0,\ldots,1),\ldots,(1,\ldots,1)\}| = 2^n$\\
Blockchiffre = Permuation der $2^n$ Bl�cke.\\
$(2^n)!$ Blockchiffre\\
Wenn alle verwendet werden:\\
Schl�ssel = Permuation der $2^n$ Bl�cke\\
$(x_{1,1},\ldots,x_{1,n},x_{2,1},\ldots,x_{2,n},\ldots)$ \ \ \fbox{$n\cdot 2^n$ Bit}\\
Zur Speicherung eines Schl�ssels werden $n \cdot 2^n$ Bit ben�tigt.\\
Zum Beispiel:\\
$n=64, \ 64 \cdot 2^{64}=2^{70}\approx$ 1 ZetaByte $\approx$ 1 Milliarde Festplatten � 1 TB\\
\textbf{Illusional!}\\
Konsequenz:\ Verwende Verfahren, wo nur ein kleiner Teil der Permutation als Schl�ssel verwendet wird und so sich die Schl�ssel dann in k�rzerer Fom darstellt.
%
% Ende zweiter Vorlesung
%