\chapter{Signaturen, Hashfunktionen, Authentifizierung}
\section{Anforderung an digitale Signaturen}
\paragraph{Identit�tseigenschaft:}

ID des Unterzeichners des Dokuments wird sichergestellt
\paragraph{Echtheitseigenschaft:}

des signiertem Dokument

\paragraph{Verifikationseigenschaft:}

Jeder Empf�nger muss digitale Signatur verifizieren k�nnen.

\section{RSA-Signatur (vereinfachte Version)}

$A$ will Dokument $m$ signieren. \\
$A$ bestitzt �ffentlichen RSA-Schl�ssel $(n,e)$, geheimen Schl�ssel $d$.

Signatur: $m^d\ mod\ n$ sendet $(m,\ m^d\ mod\ n)$ an $B$.\\
$(m^d\ mod\ n)^e = m^{e \cdot d}\ mod\ n = m\ (mod\ n)$\\
$m < n$\\
Wenn $m^{e \cdot d}\ mod\ n = m$, dann akzeptiert $B$ die Signatur.\\
$m > n\ \ m^d\ mod\ n$ $B\ mod\ n$. Ist $m^{'}\ mod\ n = m\ mod\ n$, dann $(m^{'},m^d\ mod\ n)$ g�ltige Signatur.

\subsection{Wie lassen sich lange RSA-Signaturen vermeiden?}
\paragraph{Def:}

Sei $R$ ein endliches Alphabet.\\
\textbf{Hashfunktion} $H: \R^{*} \rightarrow R^k (k \in \N$ fest $)$ soll effizient berechenbar sein.

\section{RSA-Signatur mit HASH-Funktion}

$H$ �ffentlich bekannte Hashfunktion.\\
$A$ will Nachricht $m$ signieren.\\
Bildet $H(m)$ und signiert $H(m) : H(m)^d\ mod\ n$ sendet $(m, H(m)^d\ mod\ n)$\\
Verifikation durch $B$: $m \rightarrow H(m)$\\
$(H(m)^d\ mod\ n)^e\ mod\ n = H(m)$

\subsection{Angriffsm�glichkeiten}
\begin{description}
	\item[-] Angreifer kann $H(m)$ bestimmen wenn es ihm gelingt, $m' \neq m$ zu finden, so $(m', H(m)^d\ mod\ n)$ g�ltige Signatur von $m$ durch $A$.
	\item[-] Angreife w�hlt zuf�llig $y$ und berechnet $y^e\mod\ n = z$\\
	Gelingt es ihm, $m$ zu finden mit $H(m)=z$, dann ist $(m,y)$ g�tlige Signatur von $m$ durch $A$\\
	$H(m)\ \  y^e=H(m)$ 
\end{description}

\paragraph{Def:}

Eine \textbf{kryptographische Hashfunktion} ist eine Hashfunktion, die folgende Bedinungen erf�llt.
\begin{enumerate}
	\item $H$ ist Einwegfunktion (um Angriffe des zweiten Typs zu vermeiden)
	\item $H$ ist \textbf{schwach kollisionsresistent}, d.h. zu gegebenem $m \in R^{*}$, soll es effizient nicht m�glich sein ein $m' \neq m$, mit $H(m)=H(m')$, zu finden. (um Angriffe des ersten Typs zu vermeiden)
\end{enumerate}
Versch�rfung von 2.
\begin{enumerate}
	\item[$2'$] $H$ ist \textbf{stark kollisions resistent}, wenn es effizient nicht m�glich ist $m \neq m'$ zu finden, mit  $H(m) = H (m')$.
\end{enumerate}

Da $R^{*}$ unendlich und $\left| R^k \right|=\left|R\right|^k$ endlisch ist, existiert unendlich viele Paare $(m,m'),\ m\neq m'$ mit $H(m)=H(m')$.\\
(Bilde $\left|R\right|^k+1$ viele Hashwerte: Kollision)\\
Kollisionen lassen sich nicht vermeiden, sie sollten aber nicht schnell herstellbar sein.

\subsection{Satz: Geburtstagsparadoxon}

Ein Merkmal komme in $m$ verschiedenen Auspr�gungen vor. Jede Person besitze genau eine dieser Merkmalsauspr�gungen. Ist $c \geq \frac{1+\sqrt{1+8\cdot m \cdot \ln{2}}}{2} \approx 1.18 \sqrt{m}$, so ist die Wahrscheinlichkeit, dass unter $l$ Personen zwei die gleiche Merkmalsauspr�gung haben, mindestens $\frac{1}{2}$ (Geburtstage: $m=366, l=23$).

\paragraph{Beweis}

$l$ Personen\\
Alle M�glichkeiten $(g_1,g_2,\ldots,g_l), g_i \in \{1,\ldots,m\}$ $m^l$ M�glichkeiten.\\
Alle Merkmalauspr�gungen verschieden: $m \cdot (m-1) \cdot (m-2) \cdot \ldots \cdot (m-(l-1))$\\
Wahrscheinlichkeit, dass $l$ Personen lauter verschiedene Geburtstage haben.\\
$q = \frac{m \cdot (m-1) \cdot (m-2) \cdot \ldots \cdot (m-(l-1)}{m^l}=\prod^{l-1}_0{1 - \frac{i}{m}}$\\
Wann ist $q \leq \frac{1}{2}$?\\
$e^x \geq 1+x$\\
\[
\prod^{l-1}_0{1 - \frac{i}{m}}
\leq \prod^{l-1}_0{e^{-\frac{i}{m}}}
=e^{\prod^{l-1}_0{-\frac{i}{m}}}
=e^{- \frac{1}{m}\sum^{l-1}_0{i}}
=e^{- \frac{1}{m} \cdot \frac{l \cdot (l-1)}{2}}
\]
\[
\ln{a}\leq - \frac{1}{m} \cdot \frac{l \cdot (l-1)}{2} = - \frac{l^2-l}{2 \cdot m}
\]

\subsection{Hashfunktion}

$H(m) = H(m'), m \neq m'$\\
$H: \Z_2^* \rightarrow \Z_2^n$ ($2^n$ Hashwerte)\\
Bei Erzeugung von $~2^\frac{n}{2}$ Hashwerten ist Wahrscheinlichkeit, dass zwei gleich sind $~\frac{1}{2}$\\
$n=64 : 2^{32}$ Hashwerte $(~4 \ cdot 10^9)$ unsicher.\\
Weit verbreitet waren und sind:
\begin{itemize}
	\item[] MD5 (message digerst / Ron Rivest, 1991, 128 Bit)
	\item[] SHA-1 (Secure Hash Algorithm, NSA, 1992/1993, 160 Bit)
\end{itemize}

\section{Authentifizierung}

Nachweise bzw. �berpr�fung, dass jemand derjenige ist f�r den er sich ausgbit.\\
M�glichkeiten der Authentifizierung durch:
\begin{itemize}
	\item[] \textbf{Wissen}
	\item[] Besitz
	\item[] biometische Merkmale
\end{itemize}

g�ngiste Methode: Passwort \\
Im Allgemeinem: Passwort $w$ abgespeichert als $f(w)$ $f$ Einwegfunktion.\\
$w$ $f^n(w)=w_0 \stackrel{sicher}{\rightarrow}$ Id. �berpr�fer $f$ Einweg.
\begin{enumerate}
	\item Auth. $w_1=f^{n-1}(w) \rightarrow f(f^{n-1}(w))=w_0$ ersetzt $w_0$ durch $w_1$
	\item Auth. $w_2=f^{n-2}(w) \rightarrow \ldots$
\end{enumerate}

\fbox{Passwortsicherheit: http://www.schneier.com/crypto-gram-0701.html}

\section{Challenge-Response-Authentifizierung}
RSA-Verfahren
$A \stackrel{auth.}{\longrightarrow} B$\\
�ffentlicher Schl�ssel: $(n,e)$\\
geheimer Schl�ssel: $d$\\
$A \stackrel{Zufallszahl r}{\longleftarrow} B,\ r<n$ \texttt{Challenge}\\
$A \stackrel{r^d\ mod\ n}{\longrightarrow} B$ �berpr�ft, ob $r^{d^e}\ mod\ n = r$ \texttt{Response}\\

Damit $B$ sich sicher seien kann, dass es wirklich $A$ ist, kann $B$ so oft wie es f�r n�tig h�lt neue $r$ schicken und dadurch die Chance verringern, dass $A$ nicht $A$ ist.