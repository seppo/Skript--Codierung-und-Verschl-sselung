\chapter{Der Advanced Encryption Standard (AES)}
\section{Mathematische Methoden gebraucht fuer AES} % NEED BETTER TITLE HERE
Seit 70er Jahren gab es DES (Blockl\"ange 64 Bit, Schl\"ussell\"ange 56 Bit) \\
\\
Nachfolger des DES: Daemen, Rijmen (Belgier)\\
Rijndael-Verfahren $\rightarrow$ AES (2002 FIPS 197)\\
\\
Iterierte Blockchiffre\\
Version mit 128 Bit Block und Schl\"ussel\"ange. \\
\\
$<$BILD VON EINER RUNDE VON AES KOMMT HIER HIN$>$\\
\\
Vorbemerkung: 128-Bit Bl\"ocke werden dargestellt als:\\
$\begin{pmatrix}
	a_{01} & a_{02} & \ldots & a_{03} \\
	a_{10} & a_{11} & \ldots & a_{13}\\
	\vdots & \vdots  & \vdots & \vdots \\
	a_{30} & \ldots  & \ldots & a_{33}
\end{pmatrix}$\\
\\
Jedes $a_{ij}$ = Byte\\
128er Block $\entspricht a_{00}a_{10}a_{20}\ldots a_{01}a_{11}\ldots a_{33}$ (spaltenweise gelesen)\\
\\
endlicher K\"orper: einfachste M\"oglichkeit $\Z_p$ ($p$ Primzahl)\\
$\F_{2^8}$ K\"orper mit $2^8 = 256$ Elementen\\
\\
Menge: Polynome vom Grad $< 8$ \"uber $\Z_2$\\
$b_7x^7 + \ldots + b_1x + b_0, b_i \in \Z_2 \\
(b_7, b_6, \ldots, b_0)$ Byte\\
\\
Addition = normale Addition von Polynomen\\
Multiplikation = normale Multiplikation von Polynomen + Reduktion modulo \\
irreduzibler Polynom vom Grad 8. ($x^8 + x^4 + x^3 + x + 1$)\\
\\
Bsp.\\
$(x^7 + x + 1) \odot (x^3 + x) = x^{10} + x^8 + x^4 + x^3 + x^2 + x\\
\\x^{10} + x^8 + x^4 + x^3 + x^2 + x$ mod $x^8 + x^4 + x^3 + x + 1\\
\\
x^{10} + x^8 + x^4 + x^3 + x^2 + x \div x^8 + x^4 + x^3 + x + 1 = x^2 + 1\\
\underline{x^{10} + x^6 + x^5 + x^3 + x^2}\\
\hspace*{8.85mm} x^8 + x^6 + x^5 + x^4 + x\\
\hspace*{8.85mm}\underline{ x^8 + x^4 + x^3 + x + 1} \\
\hspace*{16mm} x^6 + x^5 + x^3 + 1 \leftarrow \\
\\
(x^7 + x + 1) \odot (x^3 + x) = x^6 + x^5 + x^3 + 1$\\
\\
In $\F_{2^8}$ hat jedes Element $\neq$ 0 ein Inverses bzgl. $\odot$:\\
\indent $g \neq 0. Ex. g^{-1} \in \F_{2^8} : g \odot g^{-1} = 1$\\
\\
Erweiterte Euklid. Algo. f\"ur Polynome: \\
$g \neq 0$ (Grad $\leq 7$) \ \ $h = x^8 + x^4 + x^3 + x + 1$ irred. \ \ ggT($g, h$) = 1 \\
\\
EEA: $u,v \in \Z_2[x] : u \cdot g + v \cdot h = 1 \\
u$ mod $h =: g^{-1} \\
g^{-1} \odot g = ((u$ mod $h) \cdot g)$ mod $h = u \cdot g$ mod $h = (1 - vh)$ mod $h = 1$ mod $h = 1$\\
\section{SubBytes-Transfer}
$S_{i-1} =
\begin{pmatrix}
	a_{01} & a_{02} & \ldots & a_{03} \\
	a_{10} & a_{11} & \ldots & a_{13}\\
	\vdots & \vdots  & \vdots & \vdots \\
	a_{30} & \ldots  & \ldots & a_{33}
\end{pmatrix} , a_{ij}$ Bytes\\
Sei $g$ eines dieser Bytes, $g = (b_7b_6\ldots b_0), b_i \in \Z_2$\\
\\
1. Schritt: \ \
Fasse $g$ als Element in $\F_{2^8}$ auf.\\
\indent Ist $g = (0, \ldots, 0)$, so lasse g unver\"andert.\\
\indent Ist $g \neq (0, \ldots, 0)$, so ersetzte $g$ durch $g^{-1}$.\\
\\
2. Schritt: \ \ Ergebnis nach Schritt 1: $\tilde{g}$ wird folgenderm. Transformiert\\
\indent $\tilde{g} \cdot A + b = \tilde{\tilde{g}}$ \ \ (affin-lin. Transformation)
($\tilde{g}$: g-schlange, $\tilde{\tilde{g}}$:g-doppel-schlange)\\
\\
$A = 
\begin{pmatrix}
	1 & 1 & 1 & 1 & 1& 0 & 0 & 0\\
	0 & 1 & 1 & 1 & 1& 1 & 0 & 0\\
	\\
	\\
\end{pmatrix} \rightarrow$ zykl. Shift der vorherigen Zeile um 1 Stelle nach rechts.\\
\\
$b = \begin{pmatrix}
	1 & 1 & 0 & 0 & 0 & 1 & 1 & 0
\end{pmatrix}$\\
\\
Shritt 1 und 2 werden kombiniert, nicht jedes mal berechnet. Alle m\"oglichen Shifts ($2^8$ viele) sind in einer 16x16 Matrix und wird per Table-Lookup nachgeschlagen.\\
$g = (b_7b_6b_5b_4b_3b_2b_1b_0) \ \  b_7b_6b_5b_4$ = 0 bis 15 (Zeile) \ $b_3b_2b_1b_0$ (Spalte)

\section{Shift Rows Transformation}
4x4-Matrix von Bytes: 
$\begin{pmatrix}
	Erste\ Zeile\ unveraendert\\
	1\ stelle\ nach\ links\ zykl.\\
	2\ stellen\ nach\ links\ zykl.\\
	3\ stellen\ nach\ links\ zykl.\\
\end{pmatrix}$

\section{Mix Columns Transformation}
4x4-Matrix, Eintr\"age als Elemente in $\F_{2^8}$ auffassen.\\
Multiplikation von links mit Matrix
(Mult. der Eintr. in $\F_{2^8}):
\begin{pmatrix}
	x & x+1 & 1 & 1\\
	1 & x & x+1 & 1\\
	1 & 1 & x & x+1\\
	x+1 & 1 & 1 & x\\
\end{pmatrix}\\
x \entspricht
\begin{pmatrix}
	0 & 0 & 0 & 0 & 0 & 0 & 1 & 0\\
\end{pmatrix}$

\section{Schl\"usselerzeugung}
Ausgangsschl\"ussel hat 128 Bit. (16er String in Hexcode)\\
\\
Schreibe als 4x4-Matrix von Bytes. 4 Spalten $w(0), w(1), w(2), w(4)$. Definiere weitere 40 Spalten \`a 4 Bytes.\\
\\
$w(i-1)$ sei schon definiert.

$ 4 \nmid i : w(i) := w(i-4) \oplus w(i-1)$ (byteweise XOR)\\
$ 4 \mid i : w(i) := w(i-4) \oplus T(w(i-1))$ (T Transformation)\\
T?\\ $w(i-1) =
\begin{pmatrix}
	a\\
	b\\
	c\\
	d\\
\end{pmatrix} , \ a,\ldots,d Bytes$\\
Wende auf $b, c, d, a$ SubBytes-Transformation an $\rightarrow e, f, g, h$ $(b \rightarrow e)$\\ 
$r(i) =
\begin{pmatrix}
	0 & 0 & 0 & 0 & 0 & 0 & 1 & 0\\
\end{pmatrix}^{\frac{(i-4)}{4}}$ Potenz. in $\F_{2^8}$\\
$T(w(i-1)) = 
\begin{pmatrix}
	e \oplus r(i)\\
	f\\
	g\\
	h\\
\end{pmatrix}$\\
Rundenschl\"ussel $K_i$: 4x4-Matrix mit Spalten $w(4i), w(4i+1), w(4i+2), w(4i+3)$\\
\\
(Nebenbemerkung: Linear heisst $f(x+y) = f(x) + f(y)$)
