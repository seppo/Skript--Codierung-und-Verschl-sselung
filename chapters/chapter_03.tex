\chapter{One-Time-Pad und perfekte Sicherheit}

Lauftextverschl�sselung\\
\\
Alphabet $\Z_k=\{0,1,\ldots,k-1\}$\\
In $\Z_k$ kann man addieren und multiplizieren mit $mod\,k$.\\
\\
Klartext $x_1,x_2,\ldots,x_n$\\
Schl�sselwort $k_1,k_2,\ldots,k_n$\\
$x_1 + k_1\, mod\, k$, $x_n + k_n\, mod\, k \leftarrow$ Chiffretext\\
\\
Mit nat�rlichsprachlichen Texten ist das Verfahren unsicher.\\
$\Z_2=\{0,1\}$, $1 \oplus 1 = 0 = 0 \oplus 0$, $0 \oplus 1 = 1 = 1 \oplus 0 \Rightarrow XOR$\\
Klartext in $\Z_2^n=\{(x_1,\ldots,x_n):x_i \in \Z_2\}$
Schl�ssel: Zufallsfolge �ber $\Z_2$ der L�nge $n$. $m$ Klartext, $k$ Zufallsfolge (beide L�nge $n$)
\[
	c=m\oplus k, (x_1,\ldots,x_n)\oplus(k_1,\ldots,k_n):=(x_1\oplus k_1,\ldots,x_n\oplus k_n)
\]
\section{One-Time-Pad}
Schl�ssel $k$ darf nur einmal verwendet werden!\\
\[
	m_1\oplus k=c_1, m_2 \oplus k=c_2,c_1\oplus c_2=m_1\oplus k \oplus m_2\oplus k=m_1\oplus m_2
\]
Wieder nur Lauftext $\rightarrow$ unsicher!\\
$m_1$ und $m_2$ l�sst sich ermitteln.\\
Zufallsfolge der L�nge $n$: eigentlich unsinniger Begriff. Da jedes Bit unabh�ngig von anderen mit Wahrscheinlichkeit $\frac{1}{2}$ erzeugt wird (Output einer bin�r symmetrischen Quelle)\\
Jede Folge der L�nge $n$ ist gleich wahrscheinlich (Wahrscheinlichkeit $\frac{1}{2} n$\\
One-Time-Pad ist perfekt sicher.
\section{Perfekte Sicherheit}
Ein Verschl�sselungsverfahren ist perfekt sicher, falls gilt: F�r jeden Klartext $m$ und jedem Chiffretext $c$ (der festen L�nge $n$)
\[
	pr(m|c)=pr(m)
\]
$pr(m|c)\rightarrow$ A-posteriori-Wahrscheinlichkeit (Wahrscheinlichkeit, dass $m$ Klartext, wenn $c$ empfangen wurde)\\
$pr(m)\rightarrow$ A-priori-Wahrscheinlichkeit\\
\\
\textbf{Beispiel:} Substitutionschiffre aus Kapitel 2.\\
$n=5, m=HALLO, pr(m)>0$\\
Ang:$c=QITUA$ wird empfangen, $LL\neq TU \rightarrow pr(m|c)=0$\\
nicht perfekt sicher.\\
\\
One-Time-Pad ist perfekt sicher. \\
(Bayes'sche Formel) $m\oplus k$\\
Jede Folge $c$ l�sst sich mit geeignetem $k$ in der Form $c=m\oplus k$ erhalten.\\
W�hle $k=m\oplus c$, $m\oplus k=m\oplus m \oplus c=c$\\
Bei gegebenem $m$ und zuf�llige gew�hlten Schl�ssel $k$ ist jeder Chiffretext gleichwertig.