\chapter{Secret Sharing}
Geheimnis wird auf mehrere Teilnehmer verteilt (Teilgeheimnisse), so dass gewisse Teilmengen der Teilnehmer das Geheimnis mit ihren Teilgeheimnissen rekonstruieren k�nnen, die anderen nicht. \\ \\
$T$ = \{ $t_1, \ldots , t_n$\}, $k < n$ \ \ (T Menge der Teilnehmer) \\ \\
Jede Teilmenge von $T$ mit mindestens $k$ Teilnehmer sollen Geheimnis rekonstruieren k�nnen, Teilmengen von $T$ mit weniger als $k$ Teilnehmer nicht.\\
\section{($k, n$) - Schwellenwertsysteme}
1979 Shamir (How to share a secret)

\subsection{Konstruktion}
Vereinbarung von gro\ss er Primzahl $p$, mindestens $p \geq n + 1$
\\
\\
$g \in \Z_p = \{0, \ldots , p-1\}$

\subsection{Verteilung der Teilgeheimnisse}
Dealer w\"ahlt zuf\"allig $a_1, \ldots, a_{k-1} \in \Z_p, a_{k-1} \neq 0, k =$ Schwelle \\
$f(x) = g + a_1x + \ldots + a_{k-1}x^{k-1} \in \Z_p[x] $ \\
$(a_1, \ldots , a_{k-1}$ h\"alt er geheim, nat\"urlich auch g)\\
\\
Dealer w\"ahlt zuf\"allig $x_1, \ldots, x_n \in \Z_p$ (paarweise verschieden). \\
Teilnehmer $t_i$ erh\"alt als Teilgeheimnis $(x_i, f(x_i))$ (Punkt auf Polynom)\\
Bei $x=0$ hast du $g$.

\subsection{Rekonstruktion(sversuch) des Geheimnisses}
$k$ Teilnehmer $(x_{i_1}, f(x_{i_1})), \ldots, (x_{i_k}, f(x_{i_k}))$ \\
Durch diese Punkte ist $f$ eindeutig bestimmt, z.B. durch Lagrange-Interpol.: \\
$f(x_{i_j}) = g_{i_j} \\
\\
f(x) = \sum_{j=1}^k g_{i_j} \cdot \frac{(x-x_{i_1}), \ldots, (x-x_{i_{j-1}})(x-x_{i_{j+1}}), \ldots, (x-x_{i_k})}
                      {(x_{i_j}-x_{i_1}), \ldots, (x_{i_j}-x_{i_{j-1}})(x_{i_j}-x_{i_{j+1}}), \ldots, ((x_{i_j}-x_{i_k})}\\
f(0) = g\\
g = \sum_{j=1}^k g_{i_j} \prod_{l=j} \frac{x_{i_l}}{(x_{i_l}-x_{i_j})}$ \\
Bei mehr als $k$ Teilnehmer selbe Ergebnis.\\
Weniger als $k$ Teilnehmer $(k')$: Anderes Polynom wegen weniger Punkte, also warscheinlich anderer $g$.\\
Erzeugen Polynom vom Grad $\leq k' - 1$ \\
F\"ur alle $k \in \Z_p$ existiert gleich viele Polynome vom Grad $\leq k'-1$
durch die vorgegebene $k'$ Punkte, die bei $h$ durch $y$-Achse gehen.